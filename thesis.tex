\documentclass[12pt,twoside]{book}
\usepackage[a4paper,width=150mm,top=25mm,bottom=20mm,headheight=14.5pt]{geometry}
\usepackage[natbib,style=apa,backend=biber,backref=true]{biblatex}
\bibliography{thesis.bib}
\usepackage{fancyhdr}
\pagestyle{fancy}
\usepackage{algorithm}% http://ctan.org/pkg/algorithm
\usepackage{algpseudocode}% http://ctan.org/pkg/algorithmicx
\usepackage{amsmath,amsfonts,amssymb}
\usepackage{caption}
\usepackage{subcaption}
\usepackage{CJKutf8}
\usepackage[T1]{fontenc}
\usepackage[acronym,symbols,nogroupskip,nonumberlist]{glossaries-extra}
\usepackage{graphicx}
\graphicspath{ {figures/} }
\DeclareGraphicsExtensions{.png,.pdf,.eps}  % use during development
% \DeclareGraphicsExtensions{.eps,.pdf,.png}  % use for final production
\usepackage[bookmarks]{hyperref}
\usepackage[utf8]{inputenc}
\usepackage{mathtools}
\usepackage{siunitx}
\usepackage{tabularx}
\usepackage{textcomp}
\usepackage{titlesec}

% SIrange using em-dash
\sisetup{range-phrase=--}

% subsubsubsections paragraphs
\setcounter{secnumdepth}{3}
\titleformat{\paragraph}
{\normalfont\normalsize\bfseries}{\theparagraph}{1em}{}
\titlespacing*{\paragraph}
{0pt}{3.25ex plus 1ex minus .2ex}{1.5ex plus .2ex}

% glossary items
\makeglossaries[\acronymtype]

\glsxtrnewsymbol[description={hydropotential}]{phi}{\ensuremath{\phi}}
\glsxtrnewsymbol[description={time}]{t}{\ensuremath{t}}
\newacronym{ConvNets}{ConvNets}{Convolutional Neural Networks}
\newacronym{DEM}{DEM}{Digital Elevation Model}
\newacronym{RES}{RES}{Radio-echo sounding}
\newacronym{SLE}{SLE}{Sea level equivalent}
%Not a Number (NaN)

\DeclareSIUnit\year{yr}

\title{
{The subglacial landscape and hydrology of Antarctica as mapped from space}\\
{\large Antarctic Research Centre, Victoria University of Wellington}\\
{\includegraphics[scale=0.5]{0_arc_logo.jpg}}
}
\author{Wei Ji Leong}
\date{\today}

\begin{document}

\begin{titlepage}
  \begin{center}

    \LARGE
    \textbf{
      The subglacial landscape and hydrology of Antarctica as mapped from space
    }

    \includegraphics[width=1.0\textwidth]{0_thesis_key_figure}

    \Large
    A thesis presented for the degree of\\
    Doctor of Philosophy

    \vspace{0.8cm}

    \textbf{Wei Ji Leong}

    \includegraphics[width=0.4\textwidth]{0_arc_logo.jpg}

    \Large
    Antarctic Research Centre, Victoria University of Wellington\\
    New Zealand\\
    \today

  \end{center}
\end{titlepage}


\chapter*{Abstract}
Abstract goes here

\chapter*{Dedication}
To Penny, who led me, so profoundly, to be

\chapter*{Declaration}
I declare that..

\chapter*{Acknowledgements}

\begin{CJK*}{UTF8}{gbsn}
每一个成功男人的背后,都有一个伟大的女人。
Every step along this journey has not been without the support of a great community behind me.
My Chinese name is 梁伟吉,each character signifying a part of who I am, my duty in this world, and the enormous privillege that has allowed me to make it to this day.

Family 粱 (Leong) comes first.
My dearest partner Penny Qiu Yue 邱乐, you have went above and beyond what a normal long distance relationship entails, keeping me company daily through good times and bad, being patient, trusting and more than I could ever express in my supposedly native broken Malaysian/Bruneian Chinese.
Above all, thank you for sharing the most precious asset there is - 时间 - time - which for that, I am forever grateful.
To my younger sister Wei Qian 伟倩 and brother Wei Tong 伟桐,growing up with you two in Brunei has taught me the importance of sharing and looking after each other, so thank you for your understanding, for all the catch up calls, and simply for being.
For dad and mum,梁耀明 and 陈碹丝,it takes a million to raise one child, let alone three, and I have lost count of the number of side jobs you've had over the years, especially the struggles in recent times, both of you deserve a special thank you and a long holiday.

To my local family in Wellington, Jaclyn and David Chai, for the nourishment you have provided over the years in the form of yummy Malaysian style vegetarian food.
It is unbelievable how fast time flies, seeing your two kids ZH and Yi Ling grow tall and mature, you are all my second family and will always be remembered.

Greatness 伟 (Wei) of teachers is next.
To my great supervisor Huw Horgan (and his amazing partner Ruschle), I sometimes wonder how a person from the tropics without a trace of snow got into studying Antarctic ice, and obviously it was because of you, from that time in 2014 studying proglacial lakes at Mount Cook national park, to my 2015 Honours project on mapping potential subglacial estuaries in Antarctica, to this very PhD thesis 2017-2021.
A very big thank you for allowing me to do my work remotely using satellites, sending me randomly on exchanges to Alaska and Austria to realize how big the world truly is, for giving me the space to develop independently, while providing top notch Yoda-like advice that takes me weeks and months to figure out.

Thank you to all my Geography teachers and mentors over the years.
In particular, to my secondary school teacher Veronica Chan at Chung Hwa Middle School, for your great style of Geography teaching with funny stories and sketches, it was weird that you didn't recommend me taking Geography in high school, but here I am now.
% To my high school teacher
To my university lecturer Mairéad de Róiste, for putting me on this journey into Geographic Information Systems (GIS), setting good examples on how to tutor the next generation of students, and also for drilling in Cartogrphic best practices.
To Karl Baker at e-Spatial, for passing on your guru level of stacking and mosaicking aerial imagery, arranging map layers to perfection, and for being a role model on how to be a respectable GIS professional.

Luck 吉 (Ji) is the last but not least.
I feel really fortunate to be able to stand on the shoulders of giants, and would like to acknowledge at least some of the many people I've met along this journey.

\paragraph{Open source}

To Leonardo Uieda, DongDong Tian 田冬冬, Liam Toney and the rest of the PyGMT team, and also to Paul Wessel for his enduring work on the Generic Mapping Tools (GMT) package over many decades, it is a pleasure to have such a great tool for producing beautiful maps, thank you for all the new features and bug fixes, keep up the good work!

\paragraph{Europe}

To Poonam Vishwas, KaiKai Xue, Yawen Tan, and all the friends I met in the 2019 Erasmus+ exchange at ZGIS, University of Salzburg, Austria, it was an awesome experience having so many like-minded people researching GIS together in Europe, an experience I will always treasure.
To Dr. Hermann Klug, thank you for warmly inviting me to Salzburg and sorting out all manners of logistics in German, and to Dr. Dirk Tiede for introducing me to Object Based Image Analysis and providing a conceptual framework for understand the place of Convolutional Neural Networks in the wider field supervised classification is GIS.

\paragraph{United States}

To Anushilan Acharya, Indraneel Kasmalkar, Whyjay Zheng, and all my friends in the 2018 McCarthy International Summer School in Glaciology, it was a great pleasure meeting you all, and I look forward to seeing some of you again at some point!
To Dr. Regine Hock, thank you for inviting me to Alaska and giving me a once in a lifetime experience of stepping on a big glacier, and to Dr. Kelly Brunt for some great lessons on the ICESat-2 laser altimetry satellite!
To the International Glaciological Society (IGS), thank you for the generous early career travel grant and the chance to present a talk at the 2019 IGS Five Decades of Radioglaciology Symposium at Stanford University.

\paragraph{New Zealand}

To Arran Whiteford, Alanna Alevropoulos-Borrill, Francesca Baldacchino, my CO522 office mates at the Antarctic Research Centre, for all the yarns about ice and other topics, I couldn't have asked for a nicer group of friendly folks to do my PhD with!
To Dao Polsiri and Michelle Dow, thank you for sorting out all sorts of paperwork from flight ticket changes to funding for trivial things, and also to Aleksandr Beliaev, for troubleshooting issues with the UNIX servers and putting up with my overuse of storage, RAM, GPU resources and all.

Finally to anyone that I left out, you are not unnoticed, this sentence is for you, and I hope you enjoy reading this thesis just as much as I do.

\end{CJK*}

\tableofcontents
\listoffigures
\listoftables

\printnoidxglossary[type=symbols,sort=use,style=long,title={List of Symbols}]
\printglossary[type=\acronymtype]

\chapter{Introduction}
% The subglacial landscape and hydrology of Antarctica as mapped from space

\section{Thesis context}

% 3/7/2020 draft intro
What is the rate at which sea level will rise and affect coastal communities around the globe?
The answer to that question lies in the amount of water locked up in the Antarctic ice sheet, and to know that, we need to understand what is happening at the subglacial bed of Antarctica.

Antarctica is a large continent, about twice the size of Australia (TODO find exact stats).
The advent of satellite remote sensing techniques in the 19??s has led to an unprecedented amount of new information from the continent.
From the Landsat missions to current generation Sentinel 2, as well as daily Planet constellation pictures, the optical image of Antarctica has never looked clearer.
Also with the radar instruments like Sentinel 1 and Cryosat 2, cloud-free monitoring of the Antarctic surface has never been easier.

Sadly, such satellite measurements are typically only able to directly observe the surface of the icy continent.
There are physical constraints on designing an Earth orbiting satellite which can penetrate through both the atmosphere and kilometres of ice (TODO cite).
It is ironic to think that we know more about the subglacial landscape of Mars's poles (TODO cite) than we do of our own planet's.

% Thesis bootcamp draft below and edited on 3/7
Enter DeepBedMap, a machine learning generated dataset of Antarctica's bed which aims to be a high resolution product that captures the realities of the all-important contact surface where ice meets rock.
This knowledge comes from the abiity to infer about Antarctica's bed properties using satellite observations of the surface, combined with what little we know about the bed at discrete sites.
It is precisely the bed which we need to understand, for the flow of glaciers and ice streams all but depends on it.

The core products will a revised map of Antarctica's bed topography and subglacial hydrology, including a revised inventory of active subglacial lakes.
Building on top of this, are higher level derivative products of bed roughness and hydraulic potential.
Taken together, these dataset products act as a foundation for accurately modelling the behaviour of ice flowing on top.

\subsection{Ice flow background}

Ice flows in 3 ways:
1) deformation of the ice due to gravity
2) sliding of the ice due to water underneath a glacier
3) deformation of the bed

The latter two processes can be collectively termed as basal slip.
While the bed topography of Antarctica remains fairly constant over short timescales, the amount of water available to slide a glacier is more dynamic.
Over the past decade, while numerous studies have focused on the importance of subglacial water on ice dynamics, such as with the draining and filling of subglacial lakes, glacial surges due to rainfall events or otherwise, fewer attention has been paid to the actual structure of the topography that exerts a drag on the ice-rock contact surface.
Part of this is due to difficulty in examining the base of the ice sheet, which requires seismic or radar surveys.
In contrast, the surface of the ice sheet is much more readily observed using satellites and other remote sensing instruments.

This leads us to our thesis statement - the main question which we seek to address.
H0 - Antarctica's bed topography does NOT have an important control on ice flow (i.e. water is more important).
H1 - Antarctica's bed topography has an important control on ice flow (i.e. we need a better bed).

To answer this question, we first require an accurate model of the bed, a Digital Elevation Model of sufficient structure that allows us to predict ice and water flow over it in a way that matches reality.
Our question is inherently a geographic one, as the bed topography of Antarctica varies over geographic space.
Current generations of Antarctic bed elevation models such as BEDMAP2 are overly smooth, as even though it is based on real observations, the data is then interpolated to a 5 km grid, and then to a 1 km grid.
That of BedMachine Antarctica generated by mass conservation is also smoothed in such a way that fails to capture the inherent sub-km scale roughness of an ice sheet's topograhy as seen in paleo-ice sheets uncovered in the modern day.

A way to overcome these limitations of having a smooth topography is by using deep learning - training a neural network on areas where we have high resolution datasets, and then using it to estimate or predict what a high resolution topography would look like on an area where we have little to no direct observational data.
Our deep learning method is inspired by inverse methods, where through knowing the surface elevation, surface velocity, surface accummulation, and any other surface observations to great detail, we can infer what the bed topography must be like.
In contrast to these methods though, we also introduce a specific adversarial loss function that penalizes overtly smooth topography, pushing our bed predictions towards that of realistic groundtruth surfaces.

With the availability of this DeepBedMap dataset, we can then start to predict what ice flow over such a rough surface is like, and how it differs from ice flow over a smooth topography.
We compare that using a Full Stokes ice flow model, which is a full physical treatment of how an ice stream flows, the three-dimensional driving forces and stresses that govern the behaviour of an ice body.
In addition, we also analyze how water may flow differently over a rough versus smooth topography, whether that has any effect on hydropotential gradients, which may influence basal water pressure at the base of ice streams and thus change ice flow.
This is crucial to get right, as whether water flows as a concentrated stream in a channel or as a distributed network has a major impact on the ice flowing above it.

An important outcome of this exercise, is to determine the basal traction component of an ice stream, that which can be separated into form drag and skin drag.
Form drag is the basal drag which is due to topography - a component which typically increases when a higher resolution topography is available.
Skin drag is the frictional force that occurs at the contact surface between the ice and bedrock, and is heavily influenced by water which acts to decrease friction.
There have been some studies that have shown skin drag to be severely overestimated in comparison with form drag on glaciers such as Pine Island Glacier, and while this is important in modelling vulnerable catchments in the Amundsen sea sector, it also leads us to consider the question on whether or not this is the case for other regions in Antarctica.

Basal traction itself is normally an inferred property of the bed, and is a standard product produced by ice sheet models where given an ice surface elevation with a known velocity, as well as a bed elevation, we can invert for what basal traction is necessary to produce a desired velocity.
What is hard to separate out, is the true contribution or division of form drag and skin drag into the basal traction parameter.

We are especially interested in the actual contribution because of time.
Water beneath an ice sheet is dynamic, and it has been shown to affect the speed of mountain glaciers, and that of outlet glaciers in Greenland where surface water has the ability to access the bed via moulins.
While no active moulins are known to be present in the interior regions of West and East Antarctica (perhaps some at the Antarctic Peninsula?), there is still a considerable volume of water trapped in subglacial lakes, that of which has been shown to drain and fill up over the satellite era.
Going back to our thesis statement, we need to figure out whether the sensitivity of the Antarctic ice sheet is due to this dynamic water field (skin drag), or if knowing the bed topography (form drag) to a higher resolution itself is sufficient.

While the study of these two components are not mutually exclusive, and may perhaps be geographically dependent, it is important to be aware of both of them.
One cannot assume that water in itself is a control on ice flow via sliding (though with subglacial water of sufficient thickness, this may be the case).
Neither can one assume that a suitably rough bed topography is enough to hold back an ice sheet without an understanding of how water can change the equations in a short time period.

One main takeaway from this thesis, is the need to properly integrate the data and methods of the various fields working on Antarctic glaciology.
As remote sensing and field collected data volumes increase, it is crucial to be able to make full use of these datasets in ice sheet models to inform our understanding of future ice sheet behaviour for sea level rise projections.
The Deep Learning method we introduce offers an exciting new independent method that can efficiently ingest multiple datasets and help us to evaluate some of the weaknesses that may lie in classical models.

The bed of Antarctica may remain out of reach for the most part, save for some areas where we have drilled into it, but one can hope that the synergy between data collectors and modellers continue to highlight weaknesses in our understanding of the Antarctic bed and channel our resources into the right place.


\section{Research Questions}

This thesis will further our knowledge on the subglacial geography of Antarctica using deep learning and remote sensing techniques.
In particular, the focus will be on mapping the subglacial topography and subglacial hydrological network of Antarctica, and the implications these features have on Antarctic ice flow.
The 3 main research questions are as follows:

1) Can we use a super-resolution convolutional neural network to create a higher spatial resolution (\SI{250}{\metre}) bed elevation map of Antarctica?

2) Where does water drain and accumulate underneath the Antarctic Ice Sheet,

3) What are the insights we gain from having better knowledge of subglacial topography and hydrology to model Antarctic ice flow?

\section{Outline}

This thesis is comprised of five chapters.

Chapter 1 establishes the context behind this research, the three research questions, and also the outline you are reading now.
It also contains a literature review of our existing knowledge on the influence of subglacial topography and subglacial hydrology.

Chapter 2 is adapted from a journal paper submitted to The Cryosphere, reformatted to fit in this thesis.
It starts with an introduction to the field of deep learning in the context of geospatial science.
The chapter then provides a detailed look into the construction of a convolutional neural network architecture to generate a super-resolution (\SI{250}{\metre}) bed elevation map of Antarctica from a low resolution (\SI{1000}{\metre}) BEDMAP2 input and other remote sensing observations of the ice surface.

Chapter 3 takes the super-resolution bed elevation map generated for Antarctica, and uses it to perform ice sheet model inversions.
In this chapter, we examine the inverted basal properties of basal friction and ice rheology, and the resulting influence on ice flow with this rougher bed.

Chapter 4 looks into the active subglacial hydrology system of Antarctica using ICESat-2 laser altimetry.
A revised and automated method for building an inventory of active subglacial lakes is described, with details on the timing of subglacial lake drainage and filling events, and the estimated volume of water exchanged.
The increased spatiotemporal resolution of these subglacial hydrological maps, combined with an improved subglacial topography, gives us a remarkable look into the drivers of Antarctic ice flow.

Chapter 5 provides a discussion on the results from answering the 3 research questions, and implications for future work.
It also presents the main conclusions of this thesis.


\chapter{DeepBedMap: a deep neural network for resolving the bed topography of Antarctica}
% DeepBedMap: A super resolution deep neural network for resolving the bed topography of Antarctica

\section{Deep Neural Networks}

\subsection{Deep Neural Networks}

An artificial neural network, very loosely based on biological neural networks, is a system comprised of neurons.
Each neuron represents a simple mathematical function that takes an input $x$ and produces some output $\hat{y}$.
Neural networks are built by combining several neurons together (TODO put in Figure), either by stacking them in parallel (width-wise), or by joining them one after another (depth-wise) as multiple hidden layers.

\begin{align}
  & z = w^{\intercal} \cdot x + b \label{eq:2.1}\\
  & a = \sigma(z) \label{eq:2.2}
\end{align}

where $z$ is the output of a linear function with transposed weights $w^\intercal$ multiplied against input vector $x$, added to a bias vector $b$.
We then obtain a single neuron's output activation $a$ by passing $z$ into the sigmoid activation function $\sigma = \frac{1}{1+e^{-x}}$.
Note that the weights $w$ and bias $b$ are adjustable parameters which will be tuned during the neural network training process described later below.

The term deep neural network is used when there is not a direct mapping between the input data $x$ and the output prediction $\hat{y}$.
In other words, we call it deep when there are two or more hidden layers in the neural network.
We show below a mathematical representation on forward propagation over a simple two-layer neural network:

\begin{equation}\label{eq:2.3}
  \begin{aligned}
    & z^{{[1]}} = W^{[1]}x + b^{[1]} \\
    & a^{[1]} = \sigma(z^{[1]})
  \end{aligned}
\end{equation}

\begin{equation}\label{eq:2.4}
  \begin{aligned}
    & z^{[2]} = W^{[2]}a^{[1]} + b^{[2]} \\
    & \hat{y} = a^{[2]} = \sigma(z^{[2]})
  \end{aligned}
\end{equation}

where Equation \eqref{eq:2.3} is simply taken from the same neuron function in Equation \eqref{eq:2.1} and \eqref{eq:2.2}, modified so that $W = w^\intercal$ and with superscripts added to indicate that they are for the first layer.
Equation \eqref{eq:2.4} follows on from equation \eqref{eq:2.3}, where the activation value from the first layer $a^{[1]}$ is multiplied by the second layer's weights $W^{[2]}$ and added to the biases $b^{[2]}$.
We then pass the intermediate value $z^{[2]}$ into the sigmoid function $\sigma$ to obtain our second layer's activation value $a^{[2]}$, and as this is our final layer, the value corresponds to our predicted value $\hat{y}$.

Earlier layers in the neural network start off as simple representations of fairly simple features deduced from the input data.
Deeper layers progressively build on these earlier layers, connecting different semantic information together to form more complex feature representations that can provide useful information to generate the desired output prediction \citep{GoodfellowDeeplearning2016}.

Initially, a neural network will almost always produce output predictions that do not match the groundtruth value $y$.
The difference between the groundtruth $y$ and predicted value $\hat{y}$ is used as the basis for training the neural network.
We do this by computing the error difference, and step backwards through the neural network, gradually updating the weights of each neuron in each layer using some calculus.
Basically, the greater the contribution of a neuron to the predicted output, the greater the adjustment to that neuron's weight.
This backward update is also termed as backpropagation \citep{RumelhartLearningrepresentationsbackpropagating1986}.

Following on from the forward propagation procedure in Equation \eqref{eq:2.3} and Equation \eqref{eq:2.4}, we now demonstrate the backpropagation procedure on a logistic regression problem (values between 0 and 1):

\begin{equation}\label{eq:2.5}
  \mathcal{L}(a,y) = -y log a - (1-y)log(1-a)
\end{equation}

where the error between the neuron's activation output $a$ and groundtruth $y$ is computed using a loss function (or cost function) on one neuron $\mathcal{L}$.
The loss value from this equation is maximed (near infinity) when the activation value $a$ and groundtruth value $y$ are very different.
For example, when $a=0$ and $y=1$, $\mathcal{L} = -1 log 0 - (1-1)log(1-0) = \inf$.
To get the neural network's activation $a$ closer to the groundtruth $y$, we thus have to minimize our loss function by modifying our tunable parameters $w$ and $b$.
The steps are as follows:

\begin{equation}\label{eq:2.6}
  \frac{\partial{\mathcal{L}}}{\partial{a}} = -\frac{y}{a} + \frac{1-y}{1-a}
\end{equation}

where we first compute the partial derivatives of our loss function $\mathcal{L}$ with respect to $a$.
Following this, we have:

\begin{equation}\label{eq:2.7}
  \begin{aligned}
    & = \sigma(z) = \frac{1}{1+e^{-z}}\\
    \frac{\partial{a}}{\partial{z}} & = \frac{e^{-z}}{(1+e^{-z})^2} \\
    & = \frac{1}{(1+e^{-z})} \cdot \frac{e^{-z}}{(1+e^{-z})} \\
    & = \frac{1}{(1+e^{-z})} \cdot \frac{1 + e^{-z} -1}{(1+e^{-z})} \\
    & = \frac{1}{(1+e^{-z})} \cdot \left( 1 - \frac{1}{(1+e^{-z})} \right) \\
    & = a(1-a)
  \end{aligned}
\end{equation}

where we differentiate activation $a$ with respective to $z$.
Using the chain rule, we find that:

\begin{equation}\label{eq:2.8}
  \frac{\partial{\mathcal{L}}}{\partial{z}} = \frac{\partial{\mathcal{L}}}{\partial{a}} \cdot \frac{\partial{a}}{\partial{z}}
\end{equation}

where the partial derivative of $\mathcal{L}$ with respect to $z$ is equivalent to the dot product of Equation \eqref{eq:2.6} and Equation \eqref{eq:2.7}.
Substituting \eqref{eq:2.6} and \eqref{eq:2.7} into Equation \eqref{eq:2.8}, we can simplify the equation:

\begin{equation}\label{eq:2.9}
  \begin{aligned}
    \frac{\partial{\mathcal{L}}}{\partial{z}} & = \left( -\frac{y}{a} + \frac{1-y}{1-a} \right) \cdot (a(1-a)) \\
    & = \left( \frac{-y(1-a)}{a(a-a)} + \frac{a(1-y)}{a(1-a)} \right) \cdot (a(1-a)) \\
    & = -y + ay + a - ay \\
    & = a - y
  \end{aligned}
\end{equation}

With this, we can then compute the partial derivatives of the loss function $\mathcal{L}$ with respect to the weights $w$ and bias $b$ using the chain rule again:

\begin{equation}\label{eq:2.10}
  \begin{aligned}
    \frac{\partial{\mathcal{L}}}{\partial{w}} & = \frac{\partial{\mathcal{L}}}{\partial{z}} \cdot \frac{\partial{z}}{\partial{w}} \\
    & = (a-y) \cdot x
  \end{aligned}
\end{equation}

\begin{equation}\label{eq:2.11}
  \begin{aligned}
    \frac{\partial{\mathcal{L}}}{\partial{b}} & = \frac{\partial{\mathcal{L}}}{\partial{z}} \cdot \frac{\partial{z}}{\partial{b}} \\
    & = (a-y) \cdot 1 \\
    & = a - y
  \end{aligned}
\end{equation}

where the first term on the right hand side $\frac{\partial{\mathcal{L}}}{\partial{z}}$ is obtained from Equation \eqref{eq:2.9} and the second term is obtained by differentiating the linear function in Equation \eqref{eq:2.1} and \eqref{eq:2.2}.
Finally, we can update the weights $w$ and bias $b$ of our neuron using gradient descent as follows:

\begin{equation}\label{eq:2.12}
  \begin{aligned}
    w & \coloneqq w - \alpha \cdot \frac{\partial{\mathcal{L}}}{\partial{w}} \\
    b & \coloneqq b - \alpha \cdot \frac{\partial{\mathcal{L}}}{\partial{b}}
  \end{aligned}
\end{equation}

where we set the new weight $w$ (or bias $b$) as the old weight $w$ (or old bias $b$) minus the dot product of the learning rate $\alpha$ and the partial derivative $\frac{\partial{\mathcal{L}}}{\partial{w}}$ (or $\frac{\partial{\mathcal{L}}}{\partial{b}})$ from Equation \eqref{eq:2.10} (or Equation \eqref{eq:2.11}).
The learning rate $\alpha$ is a hyperparameter that can be adjusted to determine the size of each update increment, and is usually set as a small floating point number (e.g. 0.01).

These forward propagation (Equations \eqref{eq:2.1} to \eqref{eq:2.4}) and backpropagation (Equations \eqref{eq:2.5} to \eqref{eq:2.12}) steps represents a single iteration for training a neural network.
With each iteration, the loss value from the loss function in Equation \eqref{eq:2.6} should gradually decrease as the weights $w$ and biases $b$ in the neurons are updated.
In other words, the neural network will be able to produce a predicted result $\hat{y}$ that more closely resembles the groundtruth $y$.
The benefit of neural networks are that they can act as universal approximators for any continuous function given certain conditions \citep{LeshnoMultilayerfeedforwardnetworks1993}.
This means that well trained neural networks are able to act as independent models to most of the mathematical models we have.

\subsection{Convolutional Neural Networks}

Convolutional neural networks (\gls{ConvNets}) have their origin in the computer vision community, and are usually used in place of standard neural networks (see Section TODO label) for working on images.
They differ from standard artificial neural networks in that kernels or filters are used in place of regular neurons \citep[see][for a review]{LeCunDeeplearning2015}.
The techniques were developed in the 1980s \citep{FukushimaNeocognitronnewalgorithm1982,LeCunBackpropagationAppliedHandwritten1989} and are commonly used in pattern recognitions tasks \citep[e.g.][]{LecunGradientbasedlearningapplied1998}.
\gls{ConvNets} became a prominent tool in the computer vision community since the AlexNet architecture \citep{KrizhevskyImageNetclassificationdeep2017} almost halved the error rate of conventional object classification approaches in the 2012 ImageNet Large Scale Visual Recognition Challenge.
Besides classification tasks, \gls{ConvNets} have also been adapted for other uses.

\subsection{Generative Adversarial Networks}

\subsection{Super Resolution}


\chapter{Chapter Three Title}
% The role of subglacial topography on Antarctic ice flow

\section{Ice Sheet Modelling with a better bed}

The bed elevation of Antarctica is an important parameter in ice sheet modelling studies, yet it is one of the most difficult to observe.
Following on from our last Chapter (TODO link), we put our super-resolution Digital Elevation Model to test with an ice sheet model to see what insights a high resolution bed has in a contemporary setting.
Specifically, this chapter will look into the basal traction parameters that are deduced from inverse methods.
This contributes to the glaciology tradition of studying the role in which basal boundary conditions at the base of an ice sheet has an influence on ice flow.

\subsection{Inverse problems}

Inverse problems are common in glaciology (and geophysics in general), and involves having to know the causes given the effects.
We are trying to deduce the state of a physical system, given an operator that acts on them to produce a set of observations we can measure.
In this context, we can define the following terms:
The model parameters $m$: These are the quantities we want to know but are unable to determine directly, though they can usually be constrained to a reasonable range.
The forward model $G$: A mathematical representation of the relevant physics that maps the model parameters to some results we can observe.
The data $d$: These are the quantities we can observe and measure with some degree of precision.

The forward problem of having model parameters $m$ and passing it into the forward model $G$ to obtain data $d$ is straightforward and well-posed.
For given model parameters $m$, there are unique data $d$.
In contrast, the inverse problem of using the data $d$ and forward model $G$ to deduce the model parameters $m$ is difficult and usually ill-posed.
This is because a specific data observation $d$ may correspond to many possible solutions for $m$, or there might not be a tractable solution, especially if $G$ represents a non-linear physical system.

An additional complication is that the inverse solution to $m$ is very sensitive to small changes in $d$ that may contain errors.
It cannot be overstated then, that the aim of inverse methods is not to find a set parameters $m$ that can fit the data $d$ exactly, but to find constraints on $m$ so that the model $G$ produces results that fits to data $d$ within acceptable error limits.
Owing to the many solutions that an inverse method usually produces, there are often additional criteria that are used to reduce the number of possible solutions.
Some of these criteria may or may not be explicitly stated, and it it necessary to be aware of these additional assumptions when dealing with inverse problems.

In general, we can broadly categorize inverse methods as falling into two categories - regularization and statistical inference \citep[see][for a review]{GudmundssonInverseMethodsGlaciology2011}.
The regularization approach attempts to find a solution for parameters $m$ that can fit to the data $d$ while being able to meet the demands of the regularization term's constraints.
The statistical inference approach is often based on the Bayes Theorem, where the posterior distribution $p(m|d)$ of the state of the physical system $p$ given the data distribution $d$ is estimated.

\begin{equation}
  p(m|d) = \frac{p(m)p(d|m)}{p(d)}
\end{equation}

Either way, most of these nonlinear inverse methods use an iterative approach to minimize or maximize some given cost function.
TODO reword from \citep{GudmundssonInverseMethodsGlaciology2011}
TODO maybe change $p$ to $x$ and $d$ to $y$ and $G$ to $f$.
With the regularization approach, the cost function to be minimized could be written as $|d|_a + |p - G(p)|_b$ for some function norms $|\cdot|_a$ and $|\cdot|_b$.
With the Bayesian inference method, we wish to maximize the posterior distribution $p(m|d)$ with respect to $d$.
Such problems are computationally intensive tasks, and most ways of effectively minimizing/maximizing the cost function require a way to estimate the gradient of the cost function with respect to the model parameters $p$.
One of the first ways of calculating the gradient efficiently uses the adjoint state method \citep{MacAyealtutorialusecontrol1993}.
Since then however, other heuristical methods \citep{Pollardsimpleinversemethod2012} and a technique adapted from electric impedance tomography TODO??? \citep{ArthernFlowspeedAntarctic2015} have also been used.

\subsection{Inverting for basal conditions}

There is a strong motivation to study the basal conditions of the Antarctic Ice Sheet using inverse methods.
Ice deformation is unable to account for the observed magnitude of fast flow along ice streams, and hence the study of basal slip is paramount to understand ice dynamics.
Using readily available surface observations of ice velocity and ice elevation, and some a priori knowledge of the bed, we can invert for the mechanical conditions at the base of the ice sheet.
Only by understanding the base of the ice sheet, are we then able to predict with some confidence how the ice sheet will behave in the future.

A common exercise in ice sheet modelling studies is the need to initialize model parameters such that they are physically consistent with known observations.
This initialization step has to take place before any prognostic runs or sensitivity analyses are undertaken by the modeller.
The parameters that govern the basal boundary conditions are the most important to know in the case of fast flowing glaciers or ice streams where ice velocity is mainly due to basal slip.
The extent of which the physics that govern the movement of ice flowing over the bed terrain is governed by a sliding law.
This sliding law relates the basal sliding velocity $u_b$ with the basal shear stress $\tau_b$ (also known as basal drag or basal traction).
As the parameters within the sliding law vary over geographic space, due to variations in water pressure, bed roughness, etc, they must be determined using inverse methods.

TODO put some equations here.

\subsection{Inverting for ice rheology}

The viscosity ice is another spatially varying parameter to be determined in ice flow models using inverse methods.
The rheology of ice is nonlinear and polycrystalline glacier ice is a viscous fluid, with its viscosity being a function of stress.
At such, ice is also known as a non-Newtownian fluid, or more speficially, a power-law fluid.
Temperature also has a strong control on ice viscosity.
We can formulate the effective viscosity of ice as follows:

\begin{equation}
  \eta = \frac{1}{2A\tau^{n-1}}
\end{equation}

where the effective viscosity $\eta$ is dependent on the rate factor $A$ and the second invariant of the deviatoric stress tensor $\tau$, with $n$ being the exponent in Glen's flow law usually taken to be 3.
The rate factor $A = A(T)$ is dependent on temperature and other parameters such as water content, impurity content and crystal size.

Englacial temperatures in ice sheets should in theory be easy to calculate, given accurate boundary conditions such as geothermal heat flux.
In practice however, such data are not easy to source and contain uncertainties too great to be relied on.
Instead, it is possible to put surface data observations into an inverse model, and let it obtain the spatial distribution of viscosity that will match up with surface ice velocity.
The relative stability of ice rheology over short time periods means that we can use our estimates to conduct prognostic ice sheet model runs for decadal scale prediction.
See also studies of inverse methods to deduce rheology applied to the Ronne Ice Shelf \citep{LarourRheologyRonneIce2005}, Larsen B Ice Shelf \citep{KhazendarLarsenIceShelf2007}, and the whole of Antarctica \citep{ArthernFlowspeedAntarctic2015}.


\chapter{Chapter Four Title}
% Active subglacial hydrology in Antarctica

\section{Subglacial Hydrology}

What is the behaviour of water underneath the ice, and what methods exist to map them.
Ultimately, how much of an influence does it actually have on Antarctic ice flow?

In Antarctica, water can be easily seen on the surface in some places, mostly close to dark, low-albedo areas like rock outcrops and blue ice regions \citep{KingslakeWidespreadmovementmeltwater2017}.
The bulk of liquid water however, lies hidden beneath the ice sheet.
Over 400 subglacial lakes have now been discovered in Antarctica, and we know from geomorphological evidence that water flows in subglacial channels underneath the ice sheet \citep{SiegertRecentadvancesunderstanding2016}.
Water in the cryospheric system is especially interesting primarily because of its fluid properties.
Compared to ice, water can flow a lot more quickly over short timescales, and ice that is in contact with water is more dynamic than it would otherwise be.

Observations of water under the ice sheet are usually done using remote sensing techniques, aided by the different transmission properties of ice, water and rock.
Surveying these features requires studying either elastic wave (e.g. seismic sounding) or electromagnetic wave (e.g. ice-penetrating radar) signals.
The waves may come from active or passive sources, and are detected from sensors deployed on the ground, in the air, or onboard of satellites in space.
We know that subglacial water can exist in three ways: 1) subglacial lakes, 2) subglacial channels, 3) subglacial aquifers \citep{ColleoniSpatiotemporalvariabilityprocesses2018}.
In 1967, the first direct documented evidence for a subglacial lake in Antarctica was obtained using airborne radio echo sounding data \citep{RobinInterpretationRadioEcho1969} from a joint programme between the UK Scott Polar Research Institute, the US National Science Foundation and the Technical University of Denmark, cumulating in the first subglacial lake inventory of 17 lakes \citep{OswaldLakesAntarcticIce1973}.
This was followed by a second inventory with 77 lakes \citep{SiegertinventoryAntarcticsubglacial1996}, and a third inventory with 145 lakes \citep{SiegertrevisedinventoryAntarctic2005}, all of which were detected using radio echo sounding techniques.

Since then, large (10+ km in diameter) `active' subglacial lakes have been detected based on vertical subglacial displacements using radar interferometry \citep{GrayEvidencesubglacialwater2005}, laser altimetry \citep{Smithinventoryactivesubglacial2009} and optical image differencing \citep{FrickerActiveSubglacialWater2007}, resulting in a fourth inventory that includes 379 subglacial lakes \citep{WrightfourthinventoryAntarctic2012}.
Subsequent discoveries have followed to bring the number to above 400 \citep[e.g.][]{WrightEvidencehydrologicalconnection2012,WrightSubglacialhydrologicalconnectivity2014,KimActivesubglaciallakes2016,RiveraSubglacialLakeCECs2015,SmithConnectedsubglaciallake2017}.

The way in which subglacial water routes itself beneath the ice is itself an important question.
Various types of subglacial pathways have been theorized over the years, ranging from fast flow in concentrated channels, to slower distributed flow spread out over a larger area (see Figure TODO).
The type of subglacial drainage system is likely to be influenced by the geology of the bedrock and the temperature profile of the ice column.
For soft bed types, water may incise into the rock to form semi-circular Nye channels or broad shaped canals, perhaps even flow along cavities, or if the rock is permeable, the water may well flow within the rock itself.
For hard bed types, water may incise instead into the ice bed, forming semi-circular Rothlisberger channels (TODO macron) or broad low channels, or perhaps flow as a thin sheet of water between the ice-rock interface.
These subglacial drainage structures are not static either, but are known to change between the two extremes of efficient and inefficient regimes over space and time that has important consequences for ice dynamics \citep{MullerVelocityfluctuationswater1973}.
However, it is important to keep in mind that the treatment of Antarctic glaciers or ice streams does differ from that of temperate glacier owing to the relative lack of input from surface meltwater.
The Antarctic subglacial water system is likely to be predominantly supplied from basal melt processes, such as from geothermal heat sources or perhaps from frictional heating under areas of fast ice flow.

Only in a handful of places has there been a comprehensive study that looked at the subglacial conditions of an Antarctic ice stream.
An area that was heavily focused on initially was the Whillans Ice Stream (formerly Ice Stream B) at the Siple Coast in West Antarctica where seismic surveys identified a water saturated, ~\SI{5}{\metre} thick porous till layer \citep{BlankenshipSeismicmeasurementsreveal1986} that could easily deform and explain the high surface velocities observed over that area \citep{AlleyDeformationtillice1986}.
Also in the vicinity at Subglacial Lake Whillans, ice penetrating radar \citep{ChristiansonSubglacialLakeWhillans2012} and active seismic \citep{HorganSubglacialLakeWhillans2012} surveys were conducted to constrain the thickness of the ice and water bodies as part of the Whillans Ice Stream Subglacial Access Research Drilling (WISSARD) project \citep{TulaczykWISSARDSubglacialLake2014}.

% TODO Debate about bed topography vs water
Indeed, while there is little question as to the importance of the bed, there has been considerable debate on the influence of water versus topographic controls on the flow of ice.
Water appears to be a clear interest in first decade of the 2000s.
On one hand, \citet{BellLargesubglaciallakes2007} found the onset of rapid flow at the downslope margins of four Recovery subglacial lakes...?

However, \citep{WinsborrowWhatcontrolslocation2010} says that topographic focusing is more important that meltwater/soft beds.

On one hand, \citet{SmithConnectedsubglaciallake2017} suggest that extra water had little or no influence on the speed of the lower trunk of Thwaites Glacier.

\citet{DiezBasalSettingsControl2018} ?Recovery/Slessor/Bailey Region!! Recovery Glacier topographically controlled in downstream area, upstream controlled by basal water?!
\citep{WrightSubglacialhydrologicalconnectivity2014,StearnsIncreasedflowspeed2008} Byrd glacier speedup from glacial lake drainage


% TODO how to detect?
% hydropotential formulation
% Estimate subglacial water routes https://www.mathworks.com/matlabcentral/fileexchange/55352-how-to-estimate-subglacial-water-routes
\subsection{Hydropotential}

Hydropotential (or hydraulic potential) refers to the static energy of water available at a particular time and place, and is also occassionally referred to in the literature as hydrostatic pressure.
Its calculation is basically a function of the amount of pressure exerted on a water body, located at a particular elevation relative to a reference datum.
By calculating hydropotential over a spatial surface, we can then derive the hydropotential gradients which provides us with a measure of the direction and tendency of water to flow if suitable conduits exist in its path.
Following the methods of \citet{ShreveMovementWaterGlaciers1972}, we calculate basal hydropotential at the ice-rock interface as follows:

\begin{equation}\label{eq:4.1}
  \phi = \phi_0 + p_w + \rho_wgz_b
\end{equation}

where the hydropotential at the base of the ice \gls{phi} is equal to an arbitrary constant $\phi_0$ plus pressure due to water $p_w$ plus the elevation potential term $\rho_wgz_b$.
The elevation potential term is made up of the density of water $\rho_w$ multiplied by the gravitational acceleration term $g$ multiplied by the bed elevation $z_b$.
If we assume no ice deformation, water pressure $p_w$ at the subglacial ice-rock interface can be approximated simply as the pressure induced by the overlying ice:

\begin{equation}\label{eq:4.2}
  p_w = \rho_i * g * (z_s - z_b)
\end{equation}

where the water pressure $p_w$ is equal to the density of ice $\rho_i$ multiplied by the gravitational acceleration term $g$ multiplied by the thickness of ice $z_s - z_b$ obtained from the ice surface elevation $z_s$ minus the ice bed elevation $z_b$.
Using a gravitational acceleration $g$ of $9.8ms^{-1}$, ice density $\rho_i$ of $917kgm^{-3}$, and water density $\rho_w$ of $1030kgm^{-3}$, we can substitute Equation \eqref{eq:4.2} into \eqref{eq:4.1} to obtain the following equation:

\begin{equation}\label{eq:4.3}
  \phi = 1107.4\left(\frac{917}{113} * z_s + z_b \right)
\end{equation}


% Started 18 July 2020
\subsection{Active Subglacial Lakes}

The volume of water in subglacial lakes can change rapidly over short periods of times, happenning in a matter of days or weeks, and we call these active subglacial lakes.
As these dynamic changes in water volume typically results in changes in the ice elevation surface, we can detect active subglacial lakes more easily using satellite-based techniques, in contrast to passive subglacial lakes that can only be detected via ground-based radar and seismic surveys.


\subsubsection{Measuring Elevation (h) over the Antactic Ice Sheet}

Several satellite sensors exist to measure ice surface height, such as laser and radar altimeters, or optical photogrammetry.
A comparison of 3 main methods of satellite-based is given below:

\begin{enumerate}
  \item Laser altimeter:
  \begin{itemize}
    \item Most direct measurement of ice surface, no firn penetration
    \item Uses green 532nm wavelength light
    \item Measurements may need to be corrected by slope
    \item Affected by cloud cover
    \item E.g. ICESat-2 reaches 88°S, ICESat-1 reaches 86°S
  \end{itemize}

  \item Radar altimeter:
  \begin{itemize}
    \item Direct measurement of ice surface, but need to be corrected for firn penetration
    \item Uses radar (X band)
    \item Unaffected by cloud cover
    \item E.g. Cryosat-2 reaches 88°S
  \end{itemize}

  \item Photogrammetry:
  \begin{itemize}
    \item Somewhat direct measurement of ice surface height
    \item Requires multiple passes over the same location to derive the parallax measurement
    \item Uses optical sensors (e.g Red, Green, Blue, Infrared bands)
    \item Requires high radiometric resolution over ice sheets that are mostly white
    \item Affected by cloud cover
    \item High temporal resolution compared to altimeters
    \item E.g. Landsat-8 reaches reaches 8?°S TODO
  \end{itemize}
\end{enumerate}

We will focus on laser altimeters, specifically the ICESat-2/ATLAS satellite sensor, as it is the most direct method of obtaining ice surface height information.
Satellite altimeters work by beaming an electromagnetic wave pulse down to the Earth surface, where it is reflected off the ice surface and goes back to the satellite sensor.
The time it takes from when the beam is sent out to when it is recorded again by the sensor allows us to measure the distance between the satellite and the ice surface:

\begin{equation}\label{eq:4.4}
  d = \frac{t * c}{2}
\end{equation}

where multiplying the speed of light $c$ by the time \gls{t} taken, and dividing by two, gives the distance $d$ from the satellite to the ground.
The elevation of the ice surface is then given by:

\begin{equation}\label{eq:4.5}
  h = z_p - d
\end{equation}

where the ice surface elevation $h$ is given by the the satellite platform's elevation $z_p$ minus the distance from the satellite platform to the ice surface $d$.

% TODO insert illustration of satellite altimeter measurement

\subsubsection{Ice Surface Elevation Change over Time (dh)}

Ice surface elevation $h$ changes over time \gls{t} due to several processes, both on the surface of the ice, and what happens underneath at the bed.
It is measured as follows:

\begin{equation}\label{eq:4.6}
  dh = h_1 - h_0
\end{equation}

where the change in ice surface elevation $dh$ between two points in time is the difference between heights $h_1$ and $h_0$ respectively over time $t=1$ and $t=0$.
This ice surface elevation change can be broken down into two components:

\begin{equation}\label{eq:4.7}
  dh = dh_s + dh_b
\end{equation}

where the total change in ice surface elevation $dh$ is equal to the change in elevation due to surface processes $dh_s$ plus the change in elevation due to basal proccesses $dh_b$.

Our study focuses on the subglacial component, but because we are only able to directly measure ice surface changes $dh$ using satellites, it is important to recognize the different processes that lead to the total change in ice surface elevation over time.
Specifically, the magnitude of vertical elevation change, and the timespan in which this elevation change occurs are quite different between surface and basal processes.
We can thus use these different patterns in the elevation timeseries data to determine the cause of elevation changes.

On the surface (air-ice interface), snow accummulation can lead to an increase in elevation, while a decrease in elevation can come from snow compaction or mass wasting processes.
Surface proccesses are seasonally dependent, with an increase in elevation due to snow accummulation over the austral winter, and a decrease in elevation due to snow compaction in the austral summer.
In Antarctica, accummulation tends to be a slow and gradual process due the low precipitation over the continent, in the order of ~\SI{2}{\milli\metre\per\year} (TODO cite Arthern).
Mass wasting processes typically occur in fast flowing ice regions like glaciers and ice streams.
Over a year, the net contribution of these surface processes are typically in the order of centimetres to tens of centimetres a year.
Over a decade, these may amount to about a metre or less of elevation change, and we can this stable linear trend in most parts of the Antarctic continent over the satellite era (TODO cite 2020 Science paper on ice loss).

% TODO insert figure of some slowly accummulating/mass wasting region in Antarctica with time on x-axis and elevation on y-axis

On the bottom (ice-rock interface), changes in the volume of subglacial water, such as due to subglacial lake drainage or filling events, can also lead to ice surface elevation changes.
The mechanism of subglacial lake drainage and filling events are known (TODO or is it?, find citation), though predicting these events remain an elusive task.
The drainage or filling however, can take place rapidly at time scales of a few days or weeks, with ice surface elevation changes up to a few metres observed.
The pattern of sudden elevation change over a short timescale caused by subglacial water volume changes typically occurs over a limited geographical region of a few square kilometres (?TODO fact check).
This stands in contrast with the background gradual trend in elevation change caused by surface processes.
It is this anomalous pattern which is used as the basis of active subglacial lake detection.

\subsubsection{Rate of Elevation Change over Time (dh/dt)}

In order to reliably detect subglacial lake drainage or filling events, we need a metric that is robust enough to work on a given dataset of time series elevation measurements.
Ideally, the metric would be able to capture both the magnitude of vertical elevation change, and the timespan in which the elevation change has occured.
We will present and discuss two options of varying computational complexity to do this.

The magnitude of vertical elevation change can be represented directly in the form of a height range:

\begin{equation}\label{eq:4.8}
  h_{range} = h_{max} - h_{min}
\end{equation}

where the height range $h_{range}$ is equal to the maximum height $h_{max}$ minus the minimum height $h_{min}$ at a given point.
Computationally, this has $O(n)$ complexity (TODO double-check $O(n)$ notation).

However, there are a couple of disadvantages to this approach:

\begin{itemize}
  \item It is prone to outliers. For example, an anomalously high elevation caused by cloud interference can lead to an incorrect $h_{range}$ value.
  \item No sense of direction. There is no way to tell whether the $h_{range}$ value represents a surface rising or lowering.
\end{itemize}

That said, the calculation is fast to compute and scales linearly with the number of data points.
It can be used to quickly locate areas that have experienced a significant amount of elevation change.
Areas with high $h_{range}$ values can then be analyzed further using the linear regression method we describe next.

The rate of ice surface elevation change over time can calculated using an ordinary least squares formula:

\begin{equation}\label{eq:4.9}
  \frac{dh}{dt} = OLS( h_0, t_0, h_1, t_1, \dots, h_n, t_n )
\end{equation}

where the rate of elevation change over time $\frac{dh}{dt}$ is found through a linear regression of data points elevation $h_0$ at time $t_0$ until $h_n$ at time $t_n$.
The slope of the line of best fit represents $\frac{dh}{dt}$, and is typically given in metres per year.
Computationally, this has $O(\log n)$ complexity (TODO double-check $O(\log n)$ notation).
This makes it more computationally expensive than the $h_{range}$ calculation, and thus more effort is require to scale it to continent-wide analysis.

% TODO insert figure of linear regression of one ATL11 point

The benefits of this approach are:

\begin{itemize}
  \item The direction in which the ice surface elevation is changing is preserved, as we can tell whether the elevation is going up or down.
  \item More robust to outliers than $h_{range}$.
\end{itemize}

However, care should be taken when interpreting rate of elevation change $\frac{dh}{dt}$ for any given point.
Fine scale details in the elevation change may be masked, for example, if the surface were to rise at $t_1$ and fall at $t_2$, the $\frac{dh}{dt}$ value may not indicate a change.
Care should be taken to ensure the $\frac{dh}{dt}$ calculation starts and ends at about the same time in the year to account for seasonal effects.
While this method is less sensitive to outliers, it is still a good idea to remove points with anomalous height values before calculating $\frac{dh}{dt}$.
This can be done by filtering out the points based on quality assessment criteria such as the photon return signal.
Alternatively, a weighted linear regression can be used to give more weight to good data points (i.e. those less affected by clouds), and less weight to the anomalous points.

\subsubsection{Detecting Active Subglacial Lakes}

The algorithm we use to detect active subglacial lakes builds upon that of those used in the ICESat-1 era \citep{FrickerActiveSubglacialWater2007,Smithinventoryactivesubglacial2009}, updated for the ICESat-2 era.
In terms of active subglacial lake detection, the areas of improvements of the current generation ICESat-2 satellite include:

\begin{itemize}
  \item the six laser beam setup allow us to better handle anomalous heights due to cross-track slopes that was a major issue in ICESat-1.
  \item there is an order of magnitude increase in track density, and thus greater coverage of Northern areas of the Antarctica continent, notably near the coastal areas.
  \item the coverage hole over the South Pole has narrowed, with an additional 2 degrees of latitude from 86° S to 88° S now covered by ICESat-2
\end{itemize}

These new features in ICESat-2 allow us to simplify the active subglacial lake detection algorithm in \citep{Smithinventoryactivesubglacial2009} considerably in terms of less cross-track slope correction.
However, the increased data density of ICESat-2 over ICESat-1 also requires a more deliberate data engineering process.
While we continue to filter out data that are problematic due to clouds, there is now a specific intermediate step to filter out areas that are less likely to harbour active subglacial lakes.

Our algorithm is as follows:

\begin{enumerate}
  \item For an ATL11 point, calculate $h_{range}$. Keep ATL11 points that have $h_{range} > \SI{0.5}{\metre}$, drop the rest
  \item For the remaining ATL11 points, calculate $\frac{dh}{dt}$. Keep ATL11 points that have $\frac{dh}{dt} > \SI{0.5}{\metre\per\year}$, drop the rest
  \item For the remaining ATL11 points, check that there exists a spatial cluster of points with similar $\frac{dh}{dt}$ values surrounding it (1 sq km? TODO). Keep ATL11 points that are within high $\frac{dh}{dt}$ spatial clusers, drop the rest.
  \item For the remaining ATL11 points, cross correlate the areas with additional information such as ice velocity data, hydropotential calculations, image differencing, etc.
  \item Generate polygons of active subglacial lake boundaries, with information on estimated ice volume gained/lost over \gls{t} amount of time. TODO properly.
\end{enumerate}

Owing to the dynamic nature of these subglacial lake activity, wherein lakes can drain in a matter of weeks or months, we have automated the procedure to make it easier to get into more detailed analysis sooner.
The automation allows us to detect such events every season, or every month, depending on data availability.

One thing that the increased ICESat-2 spatial data density has shown us over ICESat-1 is that the distribution of active subglacial lakes over Antarctica is greater than once thought.
Instead of ~145 subglacial lakes with a combined volume of ~?? cubic metres \citep{Smithinventoryactivesubglacial2009}, we have detected ~??? subglacial lakes with a combined volume of ~?? cubic metres.
This has implications for (to be continued).


\chapter{Conclusion}
\input{chapters/05_conclusion}

\appendix
\chapter{Appendix Title}
% Appendix

\section{Hydropotential} \label{sec:hydropotential}

% hydropotential formulation
% Estimate subglacial water routes https://www.mathworks.com/matlabcentral/fileexchange/55352-how-to-estimate-subglacial-water-routes

Hydropotential (or hydrostatic pressure) refers to the static energy of water available at a particular time and place.
It is a function of the amount of pressure exerted on a water body, located at a particular elevation relative to a reference datum.
By calculating hydropotential over a spatial surface, we can then derive the hydropotential gradients which provides us with a measure of the direction and tendency of water to flow if suitable conduits exist in its path.
Following the methods of \citet{ShreveMovementWaterGlaciers1972}, basal hydropotential \gls{phi} is calculated as follows:

\begin{equation}\label{eq:4.11}
  \phi = \phi_0 + p_w + \rho_wgz_b
\end{equation}

where \gls{phi} denotes hydropotential at the base of the ice, $\phi_0$ is an arbitrary constant, $p_w$ is water pressure and $\rho_wgz_b$ is the elevation potential term.
The elevation potential term is made up of the density of water $\rho_w$ multiplied by the gravitational acceleration term $g$ multiplied by the bed elevation $z_b$.
In most cases, subglacial water pressure $p_w$ can be approximated as the pressure induced by the overlying ice (overburden pressure):

\begin{equation}\label{eq:4.12}
  p_w = \rho_i * g * (z_s - z_b)
\end{equation}

where the static water pressure $p_w$ is equal to the density of ice $\rho_i$ multiplied by the gravitational acceleration term $g$ multiplied by the thickness of ice $z_s - z_b$ obtained from the ice surface elevation $z_s$ minus the ice bed elevation $z_b$.
Using a gravitational acceleration $g$ of $9.8ms^{-1}$, ice density $\rho_i$ of $917kgm^{-3}$, and water density $\rho_w$ of $1030kgm^{-3}$, we can substitute Equation \eqref{eq:4.12} into \eqref{eq:4.11} to obtain the following equation:

\begin{equation}\label{eq:4.13}
  \phi = 1107.4\left(\frac{917}{113} * z_s + z_b \right)
\end{equation}

where the ice surface elevation $z_s$ is about $\frac{917}{113} = 8.11$ times more important than bed surface elevation $z_b$ for its effect on hydropotential $\phi$.
This is an oft-cited constant, and varies across the literature from as low as $8$ to as high as $11$ depending on what densities of water $\rho_w$ and ice $\rho_i$ are used in the calculation.

% See CarterAntarcticsubglaciallakes2017 Section 3.1, very good hydropotential equation description

% Water under the ice sheet can be detected by remote sensing techniques.
% This relies on the different transmission properties of ice, water and rock.
% Surveying these features requires studying either elastic wave (e.g. seismic sounding) or electromagnetic wave (e.g. ice-penetrating radar) signals.
% The waves may come from active or passive sources, and are detected from sensors deployed on the ground, in the air, or onboard of satellites in space.


\printbibliography

\end{document}
