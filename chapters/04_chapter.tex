% Active subglacial hydrology in Antarctica

\section{Subglacial Hydrology}

What is the behaviour of water underneath the ice, and what methods exist to map them.
Ultimately, how much of an influence does it actually have on Antarctic ice flow?

In Antarctica, water can be easily seen on the surface in some places, mostly close to dark, low-albedo areas like rock outcrops and blue ice regions \citep{KingslakeWidespreadmovementmeltwater2017}.
The bulk of liquid water however, lies hidden beneath the ice sheet.
Over 400 subglacial lakes have now been discovered in Antarctica, and we know from geomorphological evidence that water flows in subglacial channels underneath the ice sheet \citep{SiegertRecentadvancesunderstanding2016}.
Water in the cryospheric system is especially interesting primarily because of its fluid properties.
Compared to ice, water can flow a lot more quickly over short timescales, and ice that is in contact with water is more dynamic than it would otherwise be.

Observations of water under the ice sheet are usually done using remote sensing techniques, aided by the different transmission properties of ice, water and rock.
Surveying these features requires studying either elastic wave (e.g. seismic sounding) or electromagnetic wave (e.g. ice-penetrating radar) signals.
The waves may come from active or passive sources, and are detected from sensors deployed on the ground, in the air, or onboard of satellites in space.
We know that subglacial water can exist in three ways: 1) subglacial lakes, 2) subglacial channels, 3) subglacial aquifers \citep{ColleoniSpatiotemporalvariabilityprocesses2018}.
In 1967, the first direct documented evidence for a subglacial lake in Antarctica was obtained using airborne radio echo sounding data \citep{RobinInterpretationRadioEcho1969} from a joint programme between the UK Scott Polar Research Institute, the US National Science Foundation and the Technical University of Denmark, cumulating in the first subglacial lake inventory of 17 lakes \citep{OswaldLakesAntarcticIce1973}.
This was followed by a second inventory with 77 lakes \citep{SiegertinventoryAntarcticsubglacial1996}, and a third inventory with 145 lakes \citep{SiegertrevisedinventoryAntarctic2005}, all of which were detected using radio echo sounding techniques.

Since then, large (10+ km in diameter) `active' subglacial lakes have been detected based on vertical subglacial displacements using radar interferometry \citep{GrayEvidencesubglacialwater2005}, laser altimetry \citep{Smithinventoryactivesubglacial2009} and optical image differencing \citep{FrickerActiveSubglacialWater2007}, resulting in a fourth inventory that includes 379 subglacial lakes \citep{WrightfourthinventoryAntarctic2012}.
Subsequent discoveries have followed to bring the number to above 400 \citep[e.g.][]{WrightEvidencehydrologicalconnection2012,WrightSubglacialhydrologicalconnectivity2014,KimActivesubglaciallakes2016,RiveraSubglacialLakeCECs2015,SmithConnectedsubglaciallake2017}.

The way in which subglacial water routes itself beneath the ice is itself an important question.
Various types of subglacial pathways have been theorized over the years, ranging from fast flow in concentrated channels, to slower distributed flow spread out over a larger area (see Figure TODO).
The type of subglacial drainage system is likely to be influenced by the geology of the bedrock and the temperature profile of the ice column.
For soft bed types, water may incise into the rock to form semi-circular Nye channels or broad shaped canals, perhaps even flow along cavities, or if the rock is permeable, the water may well flow within the rock itself.
For hard bed types, water may incise instead into the ice bed, forming semi-circular Rothlisberger channels (TODO macron) or broad low channels, or perhaps flow as a thin sheet of water between the ice-rock interface.
These subglacial drainage structures are not static either, but are known to change between the two extremes of efficient and inefficient regimes over space and time that has important consequences for ice dynamics \citep{MullerVelocityfluctuationswater1973}.
However, it is important to keep in mind that the treatment of Antarctic glaciers or ice streams does differ from that of temperate glacier owing to the relative lack of input from surface meltwater.
The Antarctic subglacial water system is likely to be predominantly supplied from basal melt processes, such as from geothermal heat sources or perhaps from frictional heating under areas of fast ice flow.

Only in a handful of places has there been a comprehensive study that looked at the subglacial conditions of an Antarctic ice stream.
An area that was heavily focused on initially was the Whillans Ice Stream (formerly Ice Stream B) at the Siple Coast in West Antarctica where seismic surveys identified a water saturated, ~\SI{5}{\metre} thick porous till layer \citep{BlankenshipSeismicmeasurementsreveal1986} that could easily deform and explain the high surface velocities observed over that area \citep{AlleyDeformationtillice1986}.
Also in the vicinity at Subglacial Lake Whillans, ice penetrating radar \citep{ChristiansonSubglacialLakeWhillans2012} and active seismic \citep{HorganSubglacialLakeWhillans2012} surveys were conducted to constrain the thickness of the ice and water bodies as part of the Whillans Ice Stream Subglacial Access Research Drilling (WISSARD) project \citep{TulaczykWISSARDSubglacialLake2014}.

% TODO Debate about bed topography vs water
Indeed, while there is little question as to the importance of the bed, there has been considerable debate on the influence of water versus topographic controls on the flow of ice.
Water appears to be a clear interest in first decade of the 2000s.
On one hand, \citet{BellLargesubglaciallakes2007} found the onset of rapid flow at the downslope margins of four Recovery subglacial lakes...?

However, \citep{WinsborrowWhatcontrolslocation2010} says that topographic focusing is more important that meltwater/soft beds.

On one hand, \citet{SmithConnectedsubglaciallake2017} suggest that extra water had little or no influence on the speed of the lower trunk of Thwaites Glacier.

\citet{DiezBasalSettingsControl2018} ?Recovery/Slessor/Bailey Region!! Recovery Glacier topographically controlled in downstream area, upstream controlled by basal water?!
\citep{WrightSubglacialhydrologicalconnectivity2014,StearnsIncreasedflowspeed2008} Byrd glacier speedup from glacial lake drainage


% TODO how to detect?
% hydropotential formulation
% Estimate subglacial water routes https://www.mathworks.com/matlabcentral/fileexchange/55352-how-to-estimate-subglacial-water-routes
\subsection{Hydropotential}

Hydropotential (or hydraulic potential) refers to the static energy of water available at a particular time and place, and is also occassionally referred to in the literature as hydrostatic pressure.
Its calculation is basically a function of the amount of pressure exerted on a water body, located at a particular elevation relative to a reference datum.
By calculating hydropotential over a spatial surface, we can then derive the hydropotential gradients which provides us with a measure of the direction and tendency of water to flow if suitable conduits exist in its path.
Following the methods of \citet{ShreveMovementWaterGlaciers1972}, we calculate basal hydropotential at the ice-rock interface as follows:

\begin{equation}\label{eq:4.1}
  \phi = \phi_0 + p_w + \rho_wgz_b
\end{equation}

where the hydropotential at the base of the ice \gls{phi} is equal to an arbitrary constant $\phi_0$ plus pressure due to water $p_w$ plus the elevation potential term $\rho_wgz_b$.
The elevation potential term is made up of the density of water $\rho_w$ multiplied by the gravitational acceleration term $g$ multiplied by the bed elevation $z_b$.
If we assume no ice deformation, water pressure $p_w$ at the subglacial ice-rock interface can be approximated simply as the pressure induced by the overlying ice:

\begin{equation}\label{eq:4.2}
  p_w = \rho_i * g * (z_s - z_b)
\end{equation}

where the water pressure $p_w$ is equal to the density of ice $\rho_i$ multiplied by the gravitational acceleration term $g$ multiplied by the thickness of ice $z_s - z_b$ obtained from the ice surface elevation $z_s$ minus the ice bed elevation $z_b$.
Using a gravitational acceleration $g$ of $9.8ms^{-1}$, ice density $\rho_i$ of $917kgm^{-3}$, and water density $\rho_w$ of $1030kgm^{-3}$, we can substitute Equation \eqref{eq:4.2} into \eqref{eq:4.1} to obtain the following equation:

\begin{equation}\label{eq:4.3}
  \phi = 1107.4\left(\frac{917}{113} * z_s + z_b \right)
\end{equation}


% Started 18 July 2020
\subsection{Active Subglacial Lakes}

The volume of water in subglacial lakes can change rapidly over short periods of times, happenning in a matter of days or weeks, and we call these active subglacial lakes.
As these dynamic changes in water volume typically results in changes in the ice elevation surface, we can detect active subglacial lakes more easily using satellite-based techniques, in contrast to passive subglacial lakes that can only be detected via ground-based radar and seismic surveys.


\subsubsection{Measuring Elevation (h) over the Antactic Ice Sheet}

Several satellite sensors exist to measure ice surface height, such as laser and radar altimeters, or optical photogrammetry.
A comparison of 3 main methods of satellite-based is given below:

\begin{enumerate}
  \item Laser altimeter:
  \begin{itemize}
    \item Most direct measurement of ice surface, no firn penetration
    \item Uses green 532nm wavelength light
    \item Measurements may need to be corrected by slope
    \item Affected by cloud cover
    \item E.g. ICESat-2 reaches 88°S, ICESat-1 reaches 86°S
  \end{itemize}

  \item Radar altimeter:
  \begin{itemize}
    \item Direct measurement of ice surface, but need to be corrected for firn penetration
    \item Uses radar (X band)
    \item Unaffected by cloud cover
    \item E.g. Cryosat-2 reaches 88°S
  \end{itemize}

  \item Photogrammetry:
  \begin{itemize}
    \item Somewhat direct measurement of ice surface height
    \item Requires multiple passes over the same location to derive the parallax measurement
    \item Uses optical sensors (e.g Red, Green, Blue, Infrared bands)
    \item Requires high radiometric resolution over ice sheets that are mostly white
    \item Affected by cloud cover
    \item High temporal resolution compared to altimeters
    \item E.g. Landsat-8 reaches reaches 8?°S TODO
  \end{itemize}
\end{enumerate}

We will focus on laser altimeters, specifically the ICESat-2/ATLAS satellite sensor, as it is the most direct method of obtaining ice surface height information.
Satellite altimeters work by beaming an electromagnetic wave pulse down to the Earth surface, where it is reflected off the ice surface and goes back to the satellite sensor.
The time it takes from when the beam is sent out to when it is recorded again by the sensor allows us to measure the distance between the satellite and the ice surface:

\begin{equation}\label{eq:4.4}
  d = \frac{t * c}{2}
\end{equation}

where multiplying the speed of light $c$ by the time \gls{t} taken, and dividing by two, gives the distance $d$ from the satellite to the ground.
The elevation of the ice surface is then given by:

\begin{equation}\label{eq:4.5}
  h = z_p - d
\end{equation}

where the ice surface elevation $h$ is given by the the satellite platform's elevation $z_p$ minus the distance from the satellite platform to the ice surface $d$.

% TODO insert illustration of satellite altimeter measurement

\subsubsection{Ice Surface Elevation Change over Time (dh)}

Ice surface elevation $h$ changes over time \gls{t} due to several processes, both on the surface of the ice, and what happens underneath at the bed.
It is measured as follows:

\begin{equation}\label{eq:4.6}
  dh = h_1 - h_0
\end{equation}

where the change in ice surface elevation $dh$ between two points in time is the difference between heights $h_1$ and $h_0$ respectively over time $t=1$ and $t=0$.
This ice surface elevation change can be broken down into two components:

\begin{equation}\label{eq:4.7}
  dh = dh_s + dh_b
\end{equation}

where the total change in ice surface elevation $dh$ is equal to the change in elevation due to surface processes $dh_s$ plus the change in elevation due to basal proccesses $dh_b$.

Our study focuses on the subglacial component, but because we are only able to directly measure ice surface changes $dh$ using satellites, it is important to recognize the different processes that lead to the total change in ice surface elevation over time.
Specifically, the magnitude of vertical elevation change, and the timespan in which this elevation change occurs are quite different between surface and basal processes.
We can thus use these different patterns in the elevation timeseries data to determine the cause of elevation changes.

On the surface (air-ice interface), snow accummulation can lead to an increase in elevation, while a decrease in elevation can come from snow compaction or mass wasting processes.
Surface proccesses are seasonally dependent, with an increase in elevation due to snow accummulation over the austral winter, and a decrease in elevation due to snow compaction in the austral summer.
In Antarctica, accummulation tends to be a slow and gradual process due the low precipitation over the continent, in the order of ~\SI{2}{\milli\metre\per\year} (TODO cite Arthern).
Mass wasting processes typically occur in fast flowing ice regions like glaciers and ice streams.
Over a year, the net contribution of these surface processes are typically in the order of centimetres to tens of centimetres a year.
Over a decade, these may amount to about a metre or less of elevation change, and we can this stable linear trend in most parts of the Antarctic continent over the satellite era (TODO cite 2020 Science paper on ice loss).

% TODO insert figure of some slowly accummulating/mass wasting region in Antarctica with time on x-axis and elevation on y-axis

On the bottom (ice-rock interface), changes in the volume of subglacial water, such as due to subglacial lake drainage or filling events, can also lead to ice surface elevation changes.
The mechanism of subglacial lake drainage and filling events are known (TODO or is it?, find citation), though predicting these events remain an elusive task.
The drainage or filling however, can take place rapidly at time scales of a few days or weeks, with ice surface elevation changes up to a few metres observed.
The pattern of sudden elevation change over a short timescale caused by subglacial water volume changes typically occurs over a limited geographical region of a few square kilometres (?TODO fact check).
This stands in contrast with the background gradual trend in elevation change caused by surface processes.
It is this anomalous pattern which is used as the basis of active subglacial lake detection.

\subsubsection{Rate of Elevation Change over Time (dh/dt)}

In order to reliably detect subglacial lake drainage or filling events, we need a metric that is robust enough to work on a given dataset of time series elevation measurements.
Ideally, the metric would be able to capture both the magnitude of vertical elevation change, and the timespan in which the elevation change has occured.
We will present and discuss two options of varying computational complexity to do this.

The magnitude of vertical elevation change can be represented directly in the form of a height range:

\begin{equation}\label{eq:4.8}
  h_{range} = h_{max} - h_{min}
\end{equation}

where the height range $h_{range}$ is equal to the maximum height $h_{max}$ minus the minimum height $h_{min}$ at a given point.
Computationally, this has $O(n)$ complexity (TODO double-check $O(n)$ notation).

However, there are a couple of disadvantages to this approach:

\begin{itemize}
  \item It is prone to outliers. For example, an anomalously high elevation caused by cloud interference can lead to an incorrect $h_{range}$ value.
  \item No sense of direction. There is no way to tell whether the $h_{range}$ value represents a surface rising or lowering.
\end{itemize}

That said, the calculation is fast to compute and scales linearly with the number of data points.
It can be used to quickly locate areas that have experienced a significant amount of elevation change.
Areas with high $h_{range}$ values can then be analyzed further using the linear regression method we describe next.

The rate of ice surface elevation change over time can calculated using an ordinary least squares formula:

\begin{equation}\label{eq:4.9}
  \frac{dh}{dt} = OLS( h_0, t_0, h_1, t_1, \dots, h_n, t_n )
\end{equation}

where the rate of elevation change over time $\frac{dh}{dt}$ is found through a linear regression of data points elevation $h_0$ at time $t_0$ until $h_n$ at time $t_n$.
The slope of the line of best fit represents $\frac{dh}{dt}$, and is typically given in metres per year.
Computationally, this has $O(\log n)$ complexity (TODO double-check $O(\log n)$ notation).
This makes it more computationally expensive than the $h_{range}$ calculation, and thus more effort is require to scale it to continent-wide analysis.

% TODO insert figure of linear regression of one ATL11 point

The benefits of this approach are:

\begin{itemize}
  \item The direction in which the ice surface elevation is changing is preserved, as we can tell whether the elevation is going up or down.
  \item More robust to outliers than $h_{range}$.
\end{itemize}

However, care should be taken when interpreting rate of elevation change $\frac{dh}{dt}$ for any given point.
Fine scale details in the elevation change may be masked, for example, if the surface were to rise at $t_1$ and fall at $t_2$, the $\frac{dh}{dt}$ value may not indicate a change.
Care should be taken to ensure the $\frac{dh}{dt}$ calculation starts and ends at about the same time in the year to account for seasonal effects.
While this method is less sensitive to outliers, it is still a good idea to remove points with anomalous height values before calculating $\frac{dh}{dt}$.
This can be done by filtering out the points based on quality assessment criteria such as the photon return signal.
Alternatively, a weighted linear regression can be used to give more weight to good data points (i.e. those less affected by clouds), and less weight to the anomalous points.

\subsubsection{Detecting Active Subglacial Lakes}

The algorithm we use to detect active subglacial lakes builds upon that of those used in the ICESat-1 era \citep{FrickerActiveSubglacialWater2007,Smithinventoryactivesubglacial2009}, updated for the ICESat-2 era.
In terms of active subglacial lake detection, the areas of improvements of the current generation ICESat-2 satellite include:

\begin{itemize}
  \item the six laser beam setup allow us to better handle anomalous heights due to cross-track slopes that was a major issue in ICESat-1.
  \item there is an order of magnitude increase in track density, and thus greater coverage of Northern areas of the Antarctica continent, notably near the coastal areas.
  \item the coverage hole over the South Pole has narrowed, with an additional 2 degrees of latitude from 86° S to 88° S now covered by ICESat-2
\end{itemize}

These new features in ICESat-2 allow us to simplify the active subglacial lake detection algorithm in \citep{Smithinventoryactivesubglacial2009} considerably in terms of less cross-track slope correction.
However, the increased data density of ICESat-2 over ICESat-1 also requires a more deliberate data engineering process.
While we continue to filter out data that are problematic due to clouds, there is now a specific intermediate step to filter out areas that are less likely to harbour active subglacial lakes.

Our algorithm is as follows:

\begin{enumerate}
  \item For an ATL11 point, calculate $h_{range}$. Keep ATL11 points that have $h_{range} > \SI{0.5}{\metre}$, drop the rest
  \item For the remaining ATL11 points, calculate $\frac{dh}{dt}$. Keep ATL11 points that have $\frac{dh}{dt} > \SI{0.5}{\metre\per\year}$, drop the rest
  \item For the remaining ATL11 points, check that there exists a spatial cluster of points with similar $\frac{dh}{dt}$ values surrounding it (1 sq km? TODO). Keep ATL11 points that are within high $\frac{dh}{dt}$ spatial clusers, drop the rest.
  \item For the remaining ATL11 points, cross correlate the areas with additional information such as ice velocity data, hydropotential calculations, image differencing, etc.
  \item Generate polygons of active subglacial lake boundaries, with information on estimated ice volume gained/lost over \gls{t} amount of time. TODO properly.
\end{enumerate}

Owing to the dynamic nature of these subglacial lake activity, wherein lakes can drain in a matter of weeks or months, we have automated the procedure to make it easier to get into more detailed analysis sooner.
The automation allows us to detect such events every season, or every month, depending on data availability.

One thing that the increased ICESat-2 spatial data density has shown us over ICESat-1 is that the distribution of active subglacial lakes over Antarctica is greater than once thought.
Instead of ~145 subglacial lakes with a combined volume of ~?? cubic metres \citep{Smithinventoryactivesubglacial2009}, we have detected ~??? subglacial lakes with a combined volume of ~?? cubic metres.
This has implications for (to be continued).
