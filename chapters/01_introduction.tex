% The subglacial landscape and hydrology of Antarctica as mapped from space

\section{Thesis context}

% 3/7/2020 draft intro
What is the rate at which sea level will rise and affect coastal communities around the globe?
The answer to that question lies in the amount of water locked up in the Antarctic ice sheet, and to know that, we need to understand what is happening at the subglacial bed of Antarctica.

Antarctica is a large continent, about twice the size of Australia (TODO find exact stats).
The advent of satellite remote sensing techniques in the 19??s has led to an unprecedented amount of new information from the continent.
From the Landsat missions to current generation Sentinel 2, as well as daily Planet constellation pictures, the optical image of Antarctica has never looked clearer.
Also with the radar instruments like Sentinel 1 and Cryosat 2, cloud-free monitoring of the Antarctic surface has never been easier.

Sadly, such satellite measurements are typically only able to directly observe the surface of the icy continent.
There are physical constraints on designing an Earth orbiting satellite which can penetrate through both the atmosphere and kilometres of ice (TODO cite).
It is ironic to think that we know more about the subglacial landscape of Mars's poles (TODO cite) than we do of our own planet's.

% Thesis bootcamp draft below and edited on 3/7
Enter DeepBedMap, a machine learning generated dataset of Antarctica's bed which aims to be a high resolution product that captures the realities of the all-important contact surface where ice meets rock.
This knowledge comes from the abiity to infer about Antarctica's bed properties using satellite observations of the surface, combined with what little we know about the bed at discrete sites.
It is precisely the bed which we need to understand, for the flow of glaciers and ice streams all but depends on it.

The core products will a revised map of Antarctica's bed topography and subglacial hydrology, including a revised inventory of active subglacial lakes.
Building on top of this, are higher level derivative products of bed roughness and hydraulic potential.
Taken together, these dataset products act as a foundation for accurately modelling the behaviour of ice flowing on top.

\subsection{Ice flow background}

Ice flows in 3 ways:
1) deformation of the ice due to gravity
2) sliding of the ice due to water underneath a glacier
3) deformation of the bed

The latter two processes can be collectively termed as basal slip.
While the bed topography of Antarctica remains fairly constant over short timescales, the amount of water available to slide a glacier is more dynamic.
Over the past decade, while numerous studies have focused on the importance of subglacial water on ice dynamics, such as with the draining and filling of subglacial lakes, glacial surges due to rainfall events or otherwise, fewer attention has been paid to the actual structure of the topography that exerts a drag on the ice-rock contact surface.
Part of this is due to difficulty in examining the base of the ice sheet, which requires seismic or radar surveys.
In contrast, the surface of the ice sheet is much more readily observed using satellites and other remote sensing instruments.

This leads us to our thesis statement - the main question which we seek to address.
H0 - Antarctica's bed topography does NOT have an important control on ice flow (i.e. water is more important).
H1 - Antarctica's bed topography has an important control on ice flow (i.e. we need a better bed).

To answer this question, we first require an accurate model of the bed, a Digital Elevation Model of sufficient structure that allows us to predict ice and water flow over it in a way that matches reality.
Our question is inherently a geographic one, as the bed topography of Antarctica varies over geographic space.
Current generations of Antarctic bed elevation models such as BEDMAP2 are overly smooth, as even though it is based on real observations, the data is then interpolated to a 5 km grid, and then to a 1 km grid.
That of BedMachine Antarctica generated by mass conservation is also smoothed in such a way that fails to capture the inherent sub-km scale roughness of an ice sheet's topograhy as seen in paleo-ice sheets uncovered in the modern day.

A way to overcome these limitations of having a smooth topography is by using deep learning - training a neural network on areas where we have high resolution datasets, and then using it to estimate or predict what a high resolution topography would look like on an area where we have little to no direct observational data.
Our deep learning method is inspired by inverse methods, where through knowing the surface elevation, surface velocity, surface accummulation, and any other surface observations to great detail, we can infer what the bed topography must be like.
In contrast to these methods though, we also introduce a specific adversarial loss function that penalizes overtly smooth topography, pushing our bed predictions towards that of realistic groundtruth surfaces.

With the availability of this DeepBedMap dataset, we can then start to predict what ice flow over such a rough surface is like, and how it differs from ice flow over a smooth topography.
We compare that using a Full Stokes ice flow model, which is a full physical treatment of how an ice stream flows, the three-dimensional driving forces and stresses that govern the behaviour of an ice body.
In addition, we also analyze how water may flow differently over a rough versus smooth topography, whether that has any effect on hydropotential gradients, which may influence basal water pressure at the base of ice streams and thus change ice flow.
This is crucial to get right, as whether water flows as a concentrated stream in a channel or as a distributed network has a major impact on the ice flowing above it.

An important outcome of this exercise, is to determine the basal traction component of an ice stream, that which can be separated into form drag and skin drag.
Form drag is the basal drag which is due to topography - a component which typically increases when a higher resolution topography is available.
Skin drag is the frictional force that occurs at the contact surface between the ice and bedrock, and is heavily influenced by water which acts to decrease friction.
There have been some studies that have shown skin drag to be severely overestimated in comparison with form drag on glaciers such as Pine Island Glacier, and while this is important in modelling vulnerable catchments in the Amundsen sea sector, it also leads us to consider the question on whether or not this is the case for other regions in Antarctica.

Basal traction itself is normally an inferred property of the bed, and is a standard product produced by ice sheet models where given an ice surface elevation with a known velocity, as well as a bed elevation, we can invert for what basal traction is necessary to produce a desired velocity.
What is hard to separate out, is the true contribution or division of form drag and skin drag into the basal traction parameter.

We are especially interested in the actual contribution because of time.
Water beneath an ice sheet is dynamic, and it has been shown to affect the speed of mountain glaciers, and that of outlet glaciers in Greenland where surface water has the ability to access the bed via moulins.
While no active moulins are known to be present in the interior regions of West and East Antarctica (perhaps some at the Antarctic Peninsula?), there is still a considerable volume of water trapped in subglacial lakes, that of which has been shown to drain and fill up over the satellite era.
Going back to our thesis statement, we need to figure out whether the sensitivity of the Antarctic ice sheet is due to this dynamic water field (skin drag), or if knowing the bed topography (form drag) to a higher resolution itself is sufficient.

While the study of these two components are not mutually exclusive, and may perhaps be geographically dependent, it is important to be aware of both of them.
One cannot assume that water in itself is a control on ice flow via sliding (though with subglacial water of sufficient thickness, this may be the case).
Neither can one assume that a suitably rough bed topography is enough to hold back an ice sheet without an understanding of how water can change the equations in a short time period.

One main takeaway from this thesis, is the need to properly integrate the data and methods of the various fields working on Antarctic glaciology.
As remote sensing and field collected data volumes increase, it is crucial to be able to make full use of these datasets in ice sheet models to inform our understanding of future ice sheet behaviour for sea level rise projections.
The Deep Learning method we introduce offers an exciting new independent method that can efficiently ingest multiple datasets and help us to evaluate some of the weaknesses that may lie in classical models.

The bed of Antarctica may remain out of reach for the most part, save for some areas where we have drilled into it, but one can hope that the synergy between data collectors and modellers continue to highlight weaknesses in our understanding of the Antarctic bed and channel our resources into the right place.


\section{Research Questions}

This thesis will further our knowledge on the subglacial geography of Antarctica using deep learning and remote sensing techniques.
In particular, the focus will be on mapping the subglacial topography and subglacial hydrological network of Antarctica, and the implications these features have on Antarctic ice flow.
The 3 main research questions are as follows:

1) Can we use a super-resolution convolutional neural network to create a higher spatial resolution (\SI{250}{\metre}) bed elevation map of Antarctica?

2) Where does water drain and accumulate underneath the Antarctic Ice Sheet,

3) What are the insights we gain from having better knowledge of subglacial topography and hydrology to model Antarctic ice flow?

\section{Outline}

This thesis is comprised of five chapters.

Chapter 1 establishes the context behind this research, the three research questions, and also the outline you are reading now.
It also contains a literature review of our existing knowledge on the influence of subglacial topography and subglacial hydrology.

Chapter 2 is adapted from a journal paper submitted to The Cryosphere, reformatted to fit in this thesis.
It starts with an introduction to the field of deep learning in the context of geospatial science.
The chapter then provides a detailed look into the construction of a convolutional neural network architecture to generate a super-resolution (\SI{250}{\metre}) bed elevation map of Antarctica from a low resolution (\SI{1000}{\metre}) BEDMAP2 input and other remote sensing observations of the ice surface.

Chapter 3 takes the super-resolution bed elevation map generated for Antarctica, and uses it to perform ice sheet model inversions.
In this chapter, we examine the inverted basal properties of basal friction and ice rheology, and the resulting influence on ice flow with this rougher bed.

Chapter 4 looks into the active subglacial hydrology system of Antarctica using ICESat-2 laser altimetry.
A revised and automated method for building an inventory of active subglacial lakes is described, with details on the timing of subglacial lake drainage and filling events, and the estimated volume of water exchanged.
The increased spatiotemporal resolution of these subglacial hydrological maps, combined with an improved subglacial topography, gives us a remarkable look into the drivers of Antarctic ice flow.

Chapter 5 provides a discussion on the results from answering the 3 research questions, and implications for future work.
It also presents the main conclusions of this thesis.
